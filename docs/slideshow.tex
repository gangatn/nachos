\documentclass{beamer}

\usepackage{beamerthemeAmsterdam}
%\usecolortheme{lily}

% PACKAGES
\usepackage[utf8]{inputenc}
\usepackage[francais]{babel}
\usepackage{xcolor}
\usepackage{fancyvrb}
% --

\title{Projet NachOS}

\author{
  Groupe F \newline{}
  {\rule{5cm}{0.4pt}}\newline{}
  Jérémy Derdaele\newline{}
  José-Paul Dominguez\newline{}
  Timothée Frappé--Vialatoux\newline{}
  Nizâmouddine Gangat\newline{}
  {\rule{5cm}{0.4pt}}\newline{}
}

\begin{document}
\begin{frame}
  \titlepage
\end{frame}

\section*{int main(void)}
\begin{frame}
  \frametitle{Sommaire}
  \tableofcontents
\end{frame}

\section{Threads utilisateurs}
\subsection{Modèle}
\begin{frame}
  \frametitle{Modèle}
  \begin{itemize}
  \item Pas de hiérarchie entre les threads d'un espace mémoire
  \item Partage de l'espace mémoire du thread créateur
  \item ... dont la la liste threads de cet espace
  \item Identification des threads par un tid
  \end{itemize}
\end{frame}

\subsection{Primitives}
\begin{frame}
  \frametitle{Primitives}
  \begin{itemize}
    \item UserThreadCreate()
    \item UserThreadExit()
    \item UserThreadJoin()
  \end{itemize}
\end{frame}

\section{Mémoire virtuelle}
\subsection{Processus}
\begin{frame}
  \frametitle{Processus : propriétés}
  \begin{itemize}
  \item Espace d'addressage indépendants
  \item Identifié par un id (pid)
  \item Organisation hiérachique
    \begin{itemize}
    \item Un processus ``Père'' crée un processus ``Fils''
    \item Un processus ``Père'' peut attendre la terminaison d'un ``Fils''
    \end{itemize}
  \end{itemize}
\end{frame}

\begin{frame}
  \frametitle{Fork}
  \begin{itemize}
  \item On duplique entièrement la mémoire du père
  \item Certaines pages ne sont jamais modifiées
  \end{itemize}
  Approche peu efficace
\end{frame}

\begin{frame}
  \frametitle{Copy on Write}
  Lors du Fork:
  \begin{itemize}
    \item On marques toutes les pages du père et du fils en lecture seule
  \end{itemize}

  Lors d'une exception ReadOnly
  \begin{itemize}
  \item On vérifie que le processus à la permission d'écire
  \item On duplique la page concernée
  \end{itemize}

  Chaque page:
  \begin{itemize}
  \item Peut être référencée par plusieurs processus
  \item Doit être libérée uniquement quand plus aucun processus ne l'utilise
  \end{itemize}
\end{frame}

\begin{frame}
  \frametitle{Exec}
  L'espace d'adressage du processus est totalement remplacé.
\end{frame}

\begin{frame}
  \frametitle{Terminaison}
  \begin{itemize}
    \item Si un père se termine avant son fils, il est rattaché à ``init'' (pid 1)
    \item Un processus peut attendre la terminaison de son fils avec l'appel système \texttt{int WaitPid(int pid)}
  \end{itemize}

\end{frame}


\subsection{Allocation dynamique}
\begin{frame}
  \frametitle{Srbk}
  %% INSERT IMAGE OF MEMORY HERE
\end{frame}

\begin{frame}[fragile]
  \frametitle{Interpréteur Crisp}
  Un dialecte de Lisp\\
  Objectifs:
  \begin{itemize}
  \item Tester notre système sur une application ``réelle''
  \item Implanter des parties de la librairie standard du C
  \item Bénéficier d'un language de programmation (turing complet)
  \end{itemize}

  {\scriptsize
\begin{verbatim}
{define fact
    {lambda {x}
        {if { = x 0}
            1
            {* x {fact {- x 1}}}
        }
    }
}
\end{verbatim}
}
\end{frame}

\section{Système de fichier}
\subsection{Sémantique}
\begin{frame}
  \frametitle{Présentation}
  \begin{itemize}
  \item Identification des fichiers par des tags
  \item Pas de notion de hiérarchie entre les fichiers
  \item Navigation et gestion grâce aux tags
  \end{itemize}
\end{frame}

\begin{frame}
  \frametitle{Points importants}
  \begin{itemize}
  \item Trouver des fichiers doit être le plus rapide possible
    \begin{itemize}
      \item A partir d'un ensemble de tags
      \item Possibilité d'exclure facilement des tags
    \end{itemize}
  \item Minimiser la taille du système de fichier sur le disque
  \end{itemize}
\end{frame}

\begin{frame}
  \frametitle{Implantation}
  Structure :
  \begin{itemize}
  \item Groupes composés de tags et des fichiers associés
  \item Les tags référencent les groupes auxquels ils appartiennent
  \end{itemize}

  Sur le disque :
  \begin{itemize}
  \item 2 blocs réservés :
    \begin{itemize}
    \item inode mappant un bitmap des blocs libres
    \item premier tag (tags chaînés sur les blocs du disque)
    \end{itemize}
  \item Fichiers et groupes identifés par un inode
  \end{itemize}
\end{frame}

\section{Réseau}
\begin{frame}
  \frametitle{Réseau}
\end{frame}


\section*{return 0;}

\begin{frame}[fragile]
  \frametitle{Conclusion}
\begin{verbaterm}[fontsize=\scriptsize]
Questions ?

Ticks: total 869727823, idle 869133578, system 1420, user 592825
Disk I/O: reads 10298, writes 599
Console I/O: reads 354, writes 6
Paging: faults 0
Network I/O: packets received 20, sent 213

Cleaning up...
\end{verbaterm}

\end{frame}

\end{document}

\documentclass{beamer}

\usepackage{beamerthemeAmsterdam}
%\usecolortheme{lily}

% PACKAGES
\usepackage[utf8]{inputenc}
\usepackage[francais]{babel}
\usepackage{xcolor}
\usepackage{fancyvrb}
% --

\title{Projet NachOS}

\author{
  Groupe F \newline{}
  {\rule{5cm}{0.4pt}}\newline{}
  Jérémy Derdaele\newline{}
  José-Paul Dominguez\newline{}
  Timothée Frappé--Vialatoux\newline{}
  Nizâmouddine Gangat\newline{}
  {\rule{5cm}{0.4pt}}\newline{}
}

\begin{document}
\begin{frame}
  \titlepage
\end{frame}

\section*{int main(void)}
\begin{frame}
  \frametitle{Sommaire}
  \tableofcontents
\end{frame}

\section{Threads utilisateurs}
\subsection{Modèle}
\begin{frame}
  \frametitle{Modèle}
  \begin{itemize}
  \item Pas de hiérarchie entre les threads d'un espace mémoire
  \item Partage de l'espace mémoire du thread créateur
  \item ... dont la la liste threads de cet espace
  \item Identification des threads par un tid
  \end{itemize}
\end{frame}

\subsection{Primitives}
\begin{frame}
  \frametitle{Primitives}
  \begin{itemize}
    \item UserThreadCreate()
    \item UserThreadExit()
    \item UserThreadJoin()
  \end{itemize}
\end{frame}

\section{Mémoire virtuelle}
\subsection{Processus}
\begin{frame}
  \frametitle{Processus : modèle et primitives}
  \begin{itemize}
  \item Chaque processus a son propre espace d'adressage
  \item Gestion hiérarchique des processus et rattachement au pid 1
  \item Appels systèmes ``Fork'', ``Exec'', ``ForkExec''
  \item Attente d'un processus fils : ``WaitPid''
  \end{itemize}
\end{frame}

\begin{frame}
  \frametitle{Copy on write}
\end{frame}

\subsection{Allocation dynamique}
\begin{frame}
  \frametitle{Processus}
\end{frame}

\section{Système de fichier}
\subsection{Sémantique}
\begin{frame}
  \frametitle{Présentation}
  \begin{itemize}
  \item Identification des fichiers par des tags
  \item Pas de notion de hiérarchie entre les fichiers
  \item Navigation et gestion grâce aux tags
  \end{itemize}
\end{frame}

\begin{frame}
  \frametitle{Implantation}
  Structure :
  \begin{itemize}
  \item Groupes composés de tags et des fichiers associés
  \item Les tags référencent les groupes auxquels ils appartiennent
  \end{itemize}
  
  Sur le disque :
  \begin{itemize}
  \item 2 blocs réservés :
    \begin{itemize}
    \item inode mappant un bitmap des blocs libres
    \item premier tag (tags chaînés sur les blocs du disque)
    \end{itemize}
  \item Fichiers et groupes identifés par un inode
  \end{itemize}
\end{frame}

\begin{frame}
  \frametitle{Points importants}
  \begin{itemize}
  \item .
  \end{itemize}
\end{frame}

\begin{frame}
  \frametitle{Notre solution}
  \begin{itemize}
  \item Beaucoup de secteurs utilisés pour les tags et groupes
  \end{itemize}
\end{frame}

\section{Réseau}
\begin{frame}
  \frametitle{Réseau}
\end{frame}


\section*{return 0;}

\subsection{Organisation}
\begin{frame}
  \frametitle{Organisation}
\end{frame}

\subsection{Conclusion}
\begin{frame}[fragile]
  \frametitle{Conclusion}
\begin{verbaterm}[fontsize=\scriptsize]
Questions ?

Ticks: total 869727823, idle 869133578, system 1420, user 592825
Disk I/O: reads 0, writes 0
Console I/O: reads 126, writes 104
Paging: faults 0
Network I/O: packets received 0, sent 0

Cleaning up...
\end{verbaterm}

\end{frame}

\end{document}
